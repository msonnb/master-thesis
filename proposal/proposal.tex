\documentclass[12pt,a4paper]{article}

\usepackage{setspace}
\usepackage{csquotes}
\usepackage{graphicx}
\usepackage[noabbrev]{cleveref}

\title{\vspace{-3.6cm}\textsc{\large Research Proposal} \\
  Exploring ActivityPub
  interoperability
for Bluesky}

\author{Martin Sonnberger}
\date{\today}

\begin{document}

\maketitle
\onehalfspacing

\section{Introduction}

Mastodon and Bluesky are two popular microblogging social media
platforms that present themselves as open, decentralized
alternatives to traditional platforms such as X (formerly known as
Twitter). Bluesky was built as a reference application for its
underlying \emph{Authenticated Transfer Protocol}, or short ATProto, which was
designed as an alternative to the established \emph{ActivityPub}
protocol that forms the foundation of Mastodon and the Fediverse.
ActivityPub is built around an inbox/outbox model, similar to that of email.
A new user has to choose one of many available servers (instances)
when creating a new account, after which they are able to interact
with users from all participating instances. ATProto on the other
hand features multiple independent components, each of which can be
swapped out and self-hosted as long as it follows the protocol's specifications.

While Mastodon has a smaller, but loyal user base of around 10
million users\footnote{https://mastodon-analytics.com/}, Bluesky
experienced rapid growth, especially after the 2024 US presidential
election, with a current user count of around 40
million.\footnote{https://bsky.jazco.dev/stats} Since both platforms
are established in the wider market, the desire for
individuals, companies, and institutions to be represented on both
Mastodon and Bluesky has grown. However, maintaining multiple online
presences leads to additional work for content creators. To reduce
this duplicated work, bridging solutions such as Bridgy
Fed\footnote{https://fed.brid.gy/} have emerged. Bridgy Fed provides
a third party bridging service that sits in between Bluesky and
Mastodon, translating posts from one network to the other by using
synthetic bridged accounts on behalf of users that sign up and opt in
to use the service.

In this thesis, we want to take a different approach. Instead of a
separate bridging service, we want to explore making Bluesky itself
interoperable with ActivityPub and thus all Mastodon instances. We
therefore describe an ideal interoperability state that outlines how
the user experience for users on both Bluesky and Mastodon would look
like if Bluesky had full ActivityPub compatibility. We then package
the solution into three work packages with increasing complexity that
aim to achieve this ideal state. Finally, we answer the question of
how close the ideal interoperability state can be reached with
reasonable engineering effort and no protocol or implementation
changes in ActivityPub and Mastodon.

\section{Ideal state}

The goal is to achieve seamless \enquote{network dual-citizenship},
where a modified ATProto Personal Data Server (PDS) functions
simultaneously as a native Bluesky PDS and a fully compliant
ActivityPub instance. In this state, the protocol boundaries become
invisible: the user maintains a single identity that can publish,
subscribe, and moderate interactions across both the ATProto and
ActivityPub social graphs without relying on third-party bridging
services. Interactions include liking, replying, and reposting.

Users perceive a unified timeline where posts from both sources are
shown, including conversation threads involving participants from
both platforms. This interoperability extends beyond data exchange to
include governance: user safety controls, such as blocks and content
moderation flags, are federally enforced, ensuring that the user
retains full agency and protection across both networks.

\section{Work packages}

First, we define the three participating user groups:

\begin{enumerate}
  \item \textbf{Mastodon users:} These are regular users having an
    account on one of many Mastodon instances, for example
    \texttt{mastodon.social}. They will be able to interact with
    federated Bluesky users. The representative of this group is
    \textit{Mike} and their Mastodon handle is \texttt{@mike@mastodon.social}.
  \item \textbf{Federated Bluesky users:} These are Bluesky users
    with an account on our modified PDS, hosted on
    \texttt{fedisky.social}. Their accounts will be
    federated, making their profiles and content visible on Mastodon.
    They are able to interact with other Mastodon users as well as
    regular Bluesky users on other PDSs. The representative of this
    group is \textit{Frank}, and their handles are
    \texttt{@frank.fedisky.social} and \texttt{@frank@fedisky.social}
    for Bluesky and Mastodon respectively.
  \item \textbf{Regular Bluesky users:} These are Bluesky users with
    an account on some other, non-federated PDS, such as the default
    \texttt{bsky.social} PDS. They will only be able to interact with
    other Bluesky users, including federated Bluesky users. The
    representative of this group is \textit{Bob} with their Bluesky
    handle being \texttt{@bob.bsky.social}
\end{enumerate}

\subsection{Package 1: Outbound}

The first work package is about making federated Bluesky users and content
available on Mastodon. It forms itself around the following user story:

\begin{quote}
  \itshape
  \enquote{As a federated Bluesky user, I can give my handle to a
    Mastodon user. They
    can follow me, see my posts, and interact with them. I can see their
  interactions in my Bluesky notifications.}
\end{quote}

\noindent This package implements the core infrastructure required to make the
federated Bluesky user discoverable and viewable on Mastodon. In order to
achieve this, we need to make the following changes to
the Bluesky PDS:

\begin{enumerate}
  \item \textbf{Identity \& discovery:} WebFinger endpoint that maps
    the ATProto handle to an ActivityPub Actor URL and document. This
    allows discovery of the federated Bluesky user on Mastodon (i.e.
      appearing in
    search results). An additional \texttt{/.well-known/nodeinfo}
    endpoint provides information about the server and advertises
    itself as a Fediverse node.
  \item \textbf{Outbound broadcast:} When a Bluesky post is created
    on the PDS, it gets converted to an ActivityPub \textit{Note} and
    delivered  to the inboxes of all known Mastodon followers.
    Replies by federated Bluesky users on the same PDS can also be delivered via
    ActivityPub; replies from other Bluesky users will not be visible
    in Mastodon at this point. Additionally, all the federated
    Bluesky user's posts can be
    fetched from ActivityPub using a federated Bluesky user's
    \texttt{outbox} endpoint.
  \item \textbf{Inbound listening:} Accept \emph{Follow}, \emph{Like},
    and \emph{Create(Note)} (reply) activities. Follows will be stored,
    and push notifications will be created for likes and replies. The
    sequence diagram in \cref{fig:follow-sequence} illustrates what
    happens when a Mastodon user follows a federated Bluesky user.
\end{enumerate}

\begin{figure}
  \includegraphics[width=1.0\textwidth]{figures/follow}
  \caption{Sequence diagram: Mastodon user follows federated Bluesky user.}
  \label{fig:follow-sequence}
\end{figure}

\subsection{Package 2: Inbound}

For this next step, we want to bring Mastodon content into Bluesky,
by implementing the following user story:

\begin{quote}
  \itshape
  \enquote{As a federated Bluesky user, I can search for a Mastodon
    user inside my
    Bluesky app, follow them, see their posts in my timeline, and
  interact with them.}
\end{quote}

\noindent This package deals with data intake and storage. The main
challenge is that one cannot write external data into the federated
Bluesky user's
signed Merkle Search Tree (MST), the data structure that holds all
user data. Instead, we need to build a secondary storage system into
the PDS that stores all incoming ActivityPub activities. The following
steps outline the features of this work package:

\begin{enumerate}
  \item \textbf{Virtual users:} When a federated Bluesky user searches for \\
    \texttt{@mike@mastodon.social}, the PDS must fetch that Mike's
    profile and create a temporary \enquote{shadow profile} in the
    local database such that the Bluesky client can render the
    profile data in the UI. The federated Bluesky user can follow that profile,
    resulting in a \textit{Follow} activity being sent to their instance.
  \item \textbf{Shared inbox:} To scale efficiently, we implement a shared
    inbox for the whole server. Requests to this inbox
    include further information about routing activities to specific
    federated users.
  \item \textbf{Custom feed generator:} We can implement a custom
    Bluesky feed that queries the sidecar database for Mastodon posts
    and merges them with local posts, enabling a hybrid feed with
    posts from both networks.
\end{enumerate}

\subsection{Package 3: Governance \& Advanced Features}

This package refines the user experience and ensures the PDS behaves
safely within the wider ecosystem.

\begin{enumerate}
  \item \textbf{Moderation federation:} If a federated Bluesky user
    blocks an account
    in Bluesky, the PDS must broadcast a \textit{Block} activity to
    ActivityPub. On the other hand, if a PDS receives a \textit{Flag}
    (report) activity from Mastodon, it should ingest it into
    ATProto's moderation system. Similarly, content moderation (labels
    in Bluesky, content warnings in Mastodon) should be seamlessly
    translated between networks.
  \item \textbf{Account migration:} Implementing the ActivityPub
    \textit{Move} activity, allowing a federated Bluesky user to forward their
    Mastodon followers to a different Mastodon instance if they
    choose to leave our PDS.
\end{enumerate}

\end{document}