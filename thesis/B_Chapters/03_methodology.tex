\chapter{Fedisky -- An ActivityPub Federation Sidecar for Bluesky PDS}

Fedisky is a sidecar service that a Bluesky PDS operator can deploy
alongside their PDS to enable federation with the Fediverse. It acts
as a bridge between the AT Protocol and ActivityPub, allowing users
on Mastodon instances to discover, follow, and interact with users on
the Bluesky PDS. Fedisky is designed to be a
lightweight and modular service that can be easily deployed and
maintained by PDS operators, without requiring any modifications to
the PDS itself. In this chapter, we will provide an overview of the
design and implementation of Fedisky, including its architecture,
data model, key components, and operational considerations.
Fedisky's source code is available on GitHub at
\url{https://github.com/msonnb/fedisky}.

\section{Overview}

\subsection{System Context}

\begin{figure}
  \centering
  \includegraphics[width=\textwidth]{figures/c4-system-context.png}
  \caption{System Context Diagram showing how the Fedisky sidecar
  interacts with other systems and users.}
  \label{fig:system-context}
\end{figure}

\autoref{fig:system-context} shows the wider system context of
Fedisky and how it interacts with other systems and users. At the
core of the system is the PDS host, which runs both the ATProto PDS
and the Fedisky sidecar and is managed by the PDS operator. Fedisky
uses ATProto XRPC APIs to read and write records from the PDS, and
subscribe to its firehose endpoint to receive a real-time stream of
all new and updated records.

Fedisky exposes ActivityPub endpoints, which are used by
external Fediverse instances to deliver ActivityPub content to their
users and to receive incoming activities from the Fediverse. These
endpoints include a Webfinger endpoint for user discovery, an actor
endpoint for representing Bluesky users as ActivityPub actors, and an
inbox endpoint for receiving incoming activities from the Fediverse.

To fetch Bluesky records from users on other PDS instances, Fedisky
fetches records from the Bluesky AppView. In addition, Fedisky
periodically polls the ATProto Constellation
API\footnote{\url{https://constellation.microcosm.blue/}}, an
external service that aggregates and indexes backlinks across the
entire AT Protocol, allowing Fedisky to discover interactions such as
replies from users on other PDS instances, and federating them to the Fediverse.

\subsection{Technology Stack}

Fedisky is implemented in TypeScript and runs on Node.js. It stores
its data in a SQLite\footnote{\url{https://sqlite.org/}} database
using Kysely\footnote{\url{https://kysely.dev/}} as a type-safe query
builder. For federation functionality, Fedisky uses
Fedify\footnote{\url{https://fedify.dev/}}, a TypeScript library for
building ActivityPub servers.

Fedify provides a high-level API for defining ActivityPub actors,
registering dispatchers for handling incoming activities, and sending
outgoing activities to other Fediverse instances. It also handles the
underlying ActivityPub protocol primitives such as signing and
verifying HTTP requests, providing type-safe objects for Activity
Vocabulary
\footnote{\url{https://www.w3.org/TR/activitystreams-vocabulary/}}
such as \texttt{Create}, \texttt{Follow}, and \texttt{Note}. In
addition, Fedify includes scalability and reliability features such
as retry logic for failed deliveries, a message queue for processing
incoming and outgoing activities, and a KV-store for caching and
storing federation-related data such as public keys and remote actor
information.

\subsection{Subsystems}

\section{Core Flows}

\subsection{Actor and Identity Discovery}

\begin{figure}
  \centering
  \includegraphics[width=\textwidth]{figures/actor-discovery.png}
  \caption{Sequence diagram showing the actor and identity discovery
    flow when a Mastodon user tries to follow a Bluesky user on the
  PDS.}
  \label{fig:actor-discovery-sequence}
\end{figure}

When a Mastodon user tries to follow a federated Bluesky user on the
PDS, the Mastodon instance needs to discover the corresponding
ActivityPub actor for that user in order to send the follow request.
This flow is illustrated in \autoref{fig:actor-discovery-sequence}.
The Mastodon instance first queries the WebFinger endpoint with the
user's handle (e.g. \ttt{@alice@fedisky.social}) to discover the
corresponding ActivityPub actor URL. Fedisky first constructs the
ATProto handle by prepending the localpart (in this case,
\ttt{alice}) to the PDS's hostname (e.g. \ttt{fedisky.social}),
resulting in \ttt{alice.fedisky.social}. Note that the ActivityPub
handle domain and PDS domain do not have to match, but in this
example we use the same domain for simplicity. Fedisky then resolves
this handle using the PDS's \ttt{com.atproto.identity.resolveHandle}
API, which returns the corresponding DID if a matching user is found.
If a user is found, Fedisky constructs the ActivityPub actor URL
using the user's DID, resulting in
\ttt{https://fedisky.social/users/\{did\}}. This URL is returned to
the Mastodon instance in the WebFinger response, as shown in
\autoref{lst:webfinger-response}. In addition to the actor URL, the
response also includes references to the user's profile page and avatar.

\lstinputlisting[caption=WebFinger response for
  \ttt{@alice@fedisky.social},
label=lst:webfinger-response, style=jsonstyle]{listings/webfinger.json}

After receiving the WebFinger response, the Mastodon instance can
then query the actor endpoint to fetch the user's actor document,
which includes the user's profile information, public keys, and inbox
URLs. The object conforms to the Activity Vocabulary \ttt{Person}
type as defined in \cite{snellActivityVocabulary2017}. To construct
the actor document, Fedisky fetches the user's profile record form
the PDS using the \ttt{com.atproto.repo.getRecord} API. From the
profile record, Fedisky extracts the user's display name,
description, as well as avatar and banner images. Finally, Fedisky
retrieves the user's public keys from the database. If no keys exist
yet, Fedisky generates new RSA and Ed25519 key pairs and stores them
in the database.

In an effort to link accounts referring to the same identity,
\textcite{barrettBridgingIdentityAccount2025} proposes the use of
account links. In this approach, instead of relying on a separate
\enquote{meta account} that links all the user's accounts together,
accounts reference each other. Platforms can then use these references
to show highlighted links to these other accounts on different
platforms. Fedisky
follows this approach and includes the user's ATProto URI in the
\ttt{alsoKnownAs} field of the actor document, allowing Fediverse
instance to link the ActivityPub actor back to the original Bluesky
user and profile.

\subsection{Outbound: ATProto Post to ActivityPub Note}

\subsection{Inbound: ActivityPub Activity to ATProto Record}

\section{Data Model}

\subsection{Database Schema}

\subsubsection{Identity \& Cryptography}

\begin{itemize}
  \item \ttt{ap\_key\_pair} -- Stores the cryptographic key pairs
    used for HTTP signature signing. Each local PDS user gets two key
    pairs generated on first access: one RSA for compatibility with older
    ActivityPub implementations, and one Ed25519 for modern servers.
    The keys are stored as PEM-encoded strings.
  \item \ttt{ap\_bridge\_account} and
    \ttt{ap\_bluesky\_bridge\_account} -- Singleton tables that store
    the credentials for the two bridge accounts (see
    \autoref{sec:bridge-account-system}).
\end{itemize}

\subsubsection{Social Graph}

\begin{itemize}
  \item \ttt{ap\_follow} -- Records which ActivityPub actors follow
    which local PDS users. This is the core table for activity
    delivery, as it determines which users should receive which
    activities based on their follow relationships. It stores the
    follower's inbox URL and shared inbox URL for efficient delivery.
\end{itemize}

\subsubsection{Content Mapping}

\begin{itemize}
  \item \ttt{ap\_post\_mapping} -- Maps ATProto post URIs to their
    original ActivityPub Note IDs and author information. This table
    is essential for correct reply threading.
  \item \ttt{ap\_external\_reply} -- Stores external Bluesky replies
    from other PDS instances that have been federated via the
    Constellation processor. Primarily for deduplication as the
    Constellation API is polled repeatedly.
  \item \ttt{ap\_monitored\_post} -- The work queue for the
    Constellation processor, which tracks which posts need to be
    checked for new external replies.
\end{itemize}

\subsubsection{Engagement Tracking}

\begin{itemize}
  \item \ttt{ap\_like} and \ttt{ap\_repost} -- Stores incoming
    ActivityPub \ttt{Like} and \ttt{Announce} activities targeting
    local posts. The tables main purpose are to track engagements for
    display (e.g. showing like counts) and batching engagement for DM
    notifications.
\end{itemize}

\subsection{Migration Strategy}

Migrations are implemented using Kysely's built-in migration system,
which allows us to define schema changes in a type-safe manner. Each
migration is defined as a numbered TypeScript file exporting
\ttt{up()} and \ttt{down()} functions. On service startup, pending
migrations are automatically applied in order. This approach allows
us to evolve the database schema over time while ensuring data
integrity and providing a clear history of schema changes.

\section{Bridge Account System}\label{sec:bridge-account-system}

\subsection{Why Two Bridge Accounts?}

Fedisky uses two special \enquote{bridge accounts} to facilitate
bridging between the AT Protocol and ActivityPub. The first is the
Mastodon bridge account, which is hidden from users in
ActivityPub and is used to post incoming Fediverse replies as ATProto
posts on the PDS, so that they appear in the Bluesky user's thread
and can be replied to and interacted with like normal posts. Its
handle and display name can be configured by the operator, with the
default being \ttt{mastodon.\{hostname\}} and \enquote{Mastodon
Bridge} respectively.

The second one is the Bluesky bridge account, which is used to
federate replies from Bluesky users on other PDS instances to the
Fediverse. This allows users on Mastodon to see and interact with
replies from users on other PDS instances, which would otherwise be
invisible to the Fediverse. The Bluesky bridge account's default
handle is \ttt{bluesky.\{hostname\}} and display name is
\enquote{Bluesky Bridge}, again configurable by the operator via
environment variables.

\subsection{Attribution Model}

Since both bridge accounts post content on behalf on users and do not
carry any user identity in their handle or display name, we need to
ensure proper attribution of content to the original authors. For
incoming Fediverse replies posted by the Mastodon bridge account, we
include the original author's handle in the first line of the post
content, e.g. \enquote{@bob@mastodon.social replied:}, followed by
the actual reply content. Similarly, federated Bluesky replies from
third-party PDS instances sent by the Bluesky bridge account include
an attribution line with the original author's Bluesky handle, e.g.
\enquote{alice.bsky.social replied:}, followed by the reply content.
In both cases, the handle is a clickable link to the original
profile, allowing users to easily find and follow the original author
if they wish. Additionally, since Mastodon content is formatted in
HTML, we need to ensure all content is properly escaped to prevent
Cross-Site-Scripting (XSS) vulnerabilities.

\section{Inbound Federation}

\subsection{ActivityPub Endpoints}

Using Fedify together with its Express HTTP
integration\footnote{\url{https://fedify.dev/manual/integration\#express}},
we expose the following ActivityPub endpoints:

\begin{itemize}
  \item \ttt{GET /.well-known/webfinger} \\
    The WebFinger endpoint
    for user discovery. When a Mastodon instance encounters a handle
    such as \ttt{@alice@fedisky.social}, it will query this endpoint
    to discover the corresponding ActivityPub actor URL. Fedisky
    resolves the handle using the PDS's
    \ttt{com.atproto.identity.resolveHandle} API, and if a matching
    PDS user is found, constructs an ActivityPub actor URL using the
    user's DID as the unique identifier, i.e.
    \ttt{https://fedisky.social/users/\{did\}}.
  \item \ttt{GET /users/\{did\}} \\
    The actor endpoint. Returns an ActivityPub \ttt{Person} object
    representing the Bluesky user with the given DID. Includes the
    user's inbox, outbox, followers, and following URIs, as well as
    their public keys (both RSA and Ed25519), and profile information
    such as display name and avatar. It also includes a
    \ttt{alsoKnownAs} field with the user's ATProto URI, in order to
    link the ActivityPub actor back to the original PDS user and
    identity. The Mastodon bridge account does not have an
    ActivityPub actor representation.
  \item \ttt{POST /users/\{did\}/inbox} \\
    The inbox endpoint for receiving incoming activities from the
    Fediverse. This is where we receive activities such as
    \ttt{Follow} requests. When an activity is received, we verify
    the HTTP signature to ensure it is from a trusted source, and
    then dispatch it to the appropriate handler based on the activity type.
  \item \ttt{POST /inbox} \\
    A server-wide shared inbox endpoint that some Fediverse instances
    support for
    more efficient delivery. Fedisky also supports this endpoint for
    incoming activities, and dispatches them in the same way as the
    user-specific inbox.
  \item \ttt{GET /users/\{did\}/outbox} \\
    Paginated collection of posts, likes, and reposts, aggregated
    from the PDS using its \ttt{com.atproto.repo.listRecords} API.
    This allows Mastodon instance to fetch the user's content and
    engagements for display on their profile and timelines.
  \item \ttt{GET /users/\{did\}/followers} and \ttt{GET
    /users/\{did\}/following} \\
    Paginated collections of followers and following, based on the
    \ttt{ap\_follow} table. This allows Mastodon instances to display
    the user's followers and following lists, and to determine which
    users they should receive activities from.
  \item \ttt{GET /posts/\{uri\}} \\
    Endpoint for fetching a specific post by its ATProto URI, used by
    Mastodon instances to fetch the content and metadata of a post
    when displaying it or when a user clicks on a link to the post.
    Fedisky resolves the ATProto URI using the PDS's
    \ttt{com.atproto.repo.getRecord} API, and returns an ActivityPub
    Note object with the post content, author information, and any
    media attachments.
  \item \ttt{GET /nodeinfo/2.1} \\
    The NodeInfo\footnote{\url{https://nodeinfo.diaspora.software/}}
    endpoint providing metadata about the Fedisky
    instance, such as software name and version, and supported
    features. This is used by Mastodon instances to determine
    compatibility and capabilities of the Fedisky bridge.
\end{itemize}

\subsection{Inbox Processing}

\subsection{Reply Bridging}

\subsection{Post Mapping for Reply Threading}

\subsection{Engagement Notifications}

\section{Outbound Federation}

\subsection{Firehose Processor}

\subsection{Record Conversion}

\subsection{Activity Delivery}

\subsection{External Reply Discovery}

\section{Conversion Layer}

\subsection{Post Converter}

\subsection{HTML \texorpdfstring{$\leftrightarrow$}{<->} Rich Text}

\subsection{Media Handling}

\subsection{Edge Cases}

\section{Observability \& Operations}

\subsection{Wide Events Logging}

\subsection{Testing}

\subsubsection{Unit Tests}

\subsubsection{End-to-End Tests}

\subsection{Deployment}