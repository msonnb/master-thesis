\chapter{Discussion}

\section{Sidecar Architecture}

As described in the previous chapter, Fedisky is designed as a
\textit{sidecar container} as defined in
\cite[Ch.\,3]{burnsDesigningDistributedSystems2024} and runs
alongside its \textit{application container}, a Bluesky PDS. Sidecars
are a common architectural pattern, where a secondary container
provides auxiliary functionality to the primary application
container, often without the application knowing. Sidecar containers
share system resources and are scheduled to run in sync with the
application container
\cite[p.\,21]{burnsDesigningDistributedSystems2024}. This design
allows Fedisky to operate independently while
still closely integrating with the PDS.

An alternative design could have been a centralized bridging service
that bridges \textit{all} of Bluesky into the Fediverse, which is
realized by Bridgy Fed (\autoref{sec:related-work-bridgy-fed}).
Compared to a centralized bridging service, a sidecar container
running alongside a single Bluesky PDS instance, can share resources
with the PDS, most importantly the server's hostname, which allows it
to operate under the same domain and thus provide a single shared
identity for users across Bluesky and Mastodon, e.g.
\texttt{alice.fedisky.social} in Bluesky becomes
\texttt{@alice@fedisky.social} in Mastodon. Additionally, the sidecar
pattern allows for operator autonomy, where each PDS operator can
choose to deploy Fedisky independently, or not. It also reduces the
scope of the bridge, as it only needs to handle traffic for a single
PDS instance, simplifying development and maintenance.

Another approach would be to fork and modify the PDS itself, thus
integrating bridging functionality directly into the PDS. We
considered this approach initially when designing Fedisky, but chose
the sidecar pattern instead for several reasons. Forking the PDS
would significantly increase the maintenance burden and development
complexity, as one would need to keep up with upstream changes
constantly to ensure compatibility. The sidecar avoids this by only
consuming the PDS's public APIs, which are stable
contracts based on AT Protocol Lexicons \cite{HTTPReferenceBluesky}.
In addition, the sidecar pattern allows for better separation of
concerns, as all bridging logic is contained within the sidecar,
rather than the PDS being responsible for both AT Protocol and
ActivityPub compliance and therefore tightly coupling two protocol
implementations, making each harder to reason about and test in
isolation. Furthermore, the sidecar gives more deployment
flexibility, as operators can choose to start or stop the bridge
independently of the PDS. Since communication between the sidecar and
the PDS is done through HTTP APIs, the sidecar can be developed a
different technology stack, and is automatically compatible with any
PDS implementation that adheres to the AT Protocol specifications.
